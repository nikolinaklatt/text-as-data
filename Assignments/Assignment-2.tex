% Options for packages loaded elsewhere
\PassOptionsToPackage{unicode}{hyperref}
\PassOptionsToPackage{hyphens}{url}
\PassOptionsToPackage{dvipsnames,svgnames,x11names}{xcolor}
%
\documentclass[
]{article}
\usepackage{amsmath,amssymb}
\usepackage{lmodern}
\usepackage{iftex}
\ifPDFTeX
  \usepackage[T1]{fontenc}
  \usepackage[utf8]{inputenc}
  \usepackage{textcomp} % provide euro and other symbols
\else % if luatex or xetex
  \usepackage{unicode-math}
  \defaultfontfeatures{Scale=MatchLowercase}
  \defaultfontfeatures[\rmfamily]{Ligatures=TeX,Scale=1}
\fi
% Use upquote if available, for straight quotes in verbatim environments
\IfFileExists{upquote.sty}{\usepackage{upquote}}{}
\IfFileExists{microtype.sty}{% use microtype if available
  \usepackage[]{microtype}
  \UseMicrotypeSet[protrusion]{basicmath} % disable protrusion for tt fonts
}{}
\makeatletter
\@ifundefined{KOMAClassName}{% if non-KOMA class
  \IfFileExists{parskip.sty}{%
    \usepackage{parskip}
  }{% else
    \setlength{\parindent}{0pt}
    \setlength{\parskip}{6pt plus 2pt minus 1pt}}
}{% if KOMA class
  \KOMAoptions{parskip=half}}
\makeatother
\usepackage{xcolor}
\usepackage[margin=1in]{geometry}
\usepackage{graphicx}
\makeatletter
\def\maxwidth{\ifdim\Gin@nat@width>\linewidth\linewidth\else\Gin@nat@width\fi}
\def\maxheight{\ifdim\Gin@nat@height>\textheight\textheight\else\Gin@nat@height\fi}
\makeatother
% Scale images if necessary, so that they will not overflow the page
% margins by default, and it is still possible to overwrite the defaults
% using explicit options in \includegraphics[width, height, ...]{}
\setkeys{Gin}{width=\maxwidth,height=\maxheight,keepaspectratio}
% Set default figure placement to htbp
\makeatletter
\def\fps@figure{htbp}
\makeatother
\setlength{\emergencystretch}{3em} % prevent overfull lines
\providecommand{\tightlist}{%
  \setlength{\itemsep}{0pt}\setlength{\parskip}{0pt}}
\setcounter{secnumdepth}{-\maxdimen} % remove section numbering
\usepackage{booktabs}
\usepackage{xcolor}
\ifLuaTeX
  \usepackage{selnolig}  % disable illegal ligatures
\fi
\usepackage[]{natbib}
\bibliographystyle{plainnat}
\IfFileExists{bookmark.sty}{\usepackage{bookmark}}{\usepackage{hyperref}}
\IfFileExists{xurl.sty}{\usepackage{xurl}}{} % add URL line breaks if available
\urlstyle{same} % disable monospaced font for URLs
\hypersetup{
  pdftitle={Assignment 2},
  pdfauthor={Text as Data},
  colorlinks=true,
  linkcolor={Maroon},
  filecolor={Maroon},
  citecolor={Blue},
  urlcolor={blue},
  pdfcreator={LaTeX via pandoc}}

\title{Assignment 2}
\author{Text as Data}
\date{2022-11-03}

\begin{document}
\maketitle

\hypertarget{introduction}{%
\section{Introduction}\label{introduction}}

In this assignment, you are asked to use topic modelling to investigate
manifestos from the manifesto project maintained by
\href{https://manifesto-project.wzb.eu/}{WZB}. You can either use the UK
manifestos we looked at together in class, or collect your own set of
manifestos by choosing the country/countries, year/years and
party/parties you are interested in. You should produce a report which
includes your code, that addresses the following aspects of creating a
topic model, making sure to answer the questions below.

\hypertarget{data-acquisition-description-and-preparation}{%
\subsection{1. Data acquisition, description, and
preparation}\label{data-acquisition-description-and-preparation}}

Bring together a dataset from the WZB.

What years, countries and parties are included in the dataset? How many
texts do you have for each of these?

Prepare your data for topic modelling by creating a document feature
matrix. Describe the choices you make here, and comment on how these
might affect your final result.

\hypertarget{research-question}{%
\subsection{2. Research question}\label{research-question}}

Describe a research question you want to explore with topic modelling.
Comment on how answerable this is with the methods and data at your
disposal.

\hypertarget{topic-model-development}{%
\subsection{3. Topic model development}\label{topic-model-development}}

Create a topic model using your data. Explain to a non-specialist what
the topic model does. Comment on the choices you make here in terms of
hyperparameter selection and model choice. How might these affect your
results and the ability to answer your research question?

\hypertarget{topic-model-description}{%
\subsection{4. Topic model description}\label{topic-model-description}}

Describe the topic model. What topics does it contain? How are these
distributed across the data?

\hypertarget{answering-your-research-question}{%
\subsection{5. Answering your research
question}\label{answering-your-research-question}}

Use your topic model to answer your research question by showing plots
or statistical results. Discuss the implications of what you find, and
any limitations inherent in your approach. Discuss how the work could be
improved upon in future research.

  \bibliography{../presentation-resources/MyLibrary.bib}

\end{document}
